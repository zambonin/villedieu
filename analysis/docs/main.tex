\documentclass[12pt]{article}

\usepackage[T1]{fontenc}
\usepackage[a4paper, margin=1.5cm]{geometry}
\usepackage[colorlinks, urlcolor=blue, citecolor=red]{hyperref}
\usepackage[utf8]{inputenc}
\usepackage{amsmath, amsfonts, booktabs, enumitem, parskip, siunitx}

\newcommand{\euler}{\mathrm{e}}

\begin{document}

\textsc{Graduate Program in Computer Science,
  Universidade Federal de Santa Catarina} \\
\textsc{INE410104 (Design and Analysis of Algorithms)}

\textsc{Solutions to the 1\textsuperscript{st} Set of Exercises} \\
\textsc{Gustavo Zambonin, Matheus S. P. Bittencourt}

\section{Answers}

\begin{enumerate}
    \item 
    \begin{enumerate}
        \item Let the functions that represent the time complexities of algorithms $A$ and $B$ be $f(n) = n^{3}$ and $g(n) = 128n^{2}$, respectively, with $n \in \mathbb{N}$. Further, consider $t(n) = \frac{f(n)}{g(n)} \Rightarrow \frac{n}{128}$. By the ratio $t(n)$, algorithm $A$ is faster than $B$ when $n > 128$.
        \item Let the old and new computers be named $N_{1}$ and $N_{2}$, respectively, and $f(n) = 2^{n}$. The equation $t = \frac{2^{N_{1}}}{s}$ gives the time spent to solve an instance of $f(n)$ such that $N_{1}$ executes $s$ instructions per second. Analogously, let $t = \frac{2^{N_{2}}}{20s}$ represent the previous equation for the new computer. Solving for $N_{2}$ in terms of $N_{1}$:
        \begin{align*}
            \frac{2^{N_{1}}}{s} = \frac{2^{N_{2}}}{20s} \\
            2^{N_{2}} = 20 \cdot 2^{N_{1}} \\
            N_{2} = N_{1} + \log_{2} 20 \\
            N_{2} \approx N_{1} + 4.32.
        \end{align*}
        Ergo, one cannot execute $f(2n)$ and expect comparable results on $N_{2}$, since $N_{2} \ll 2 N_{1}$.
    \end{enumerate}
    \item 
    \begin{enumerate}
        \item Let $t_{1}, \dots, t_{7}$ be the number of seconds in each period of the header row and $s = 10^{10}$. The values in the table below represent approximately $n \leq f^{-1}(t_{i} s)$.
        \begin{table}[htbp]
            \renewcommand{\arraystretch}{1.2}
            \setlength{\tabcolsep}{7pt}
            \centering
            \tiny
            \begin{tabular}{l*{7}{r}}
                \toprule
                $f(n)$ / $t$ & 1 second & 1 minute & 1 hour & 1 day & 1 month & 1 year & 1 century \\ \midrule
                $\log_{2} n$ & $2^{\num{1.E+10}}$ & $2^{\num{6.E+11}}$ & $2^{\num{3.6E+13}}$ & $2^{\num{8.64E+14}}$ & $2^{\num{2.592E+16}}$ & $2^{\num{3.11E+17}}$ & $2^{\num{3.11E+19}}$ \\
                $\sqrt{n}$ & $\num{1.E+20}$ & $\num{3.6E+23}$ & $\num{1.296E+27}$ & $\num{7.465E+29}$ & $\num{6.718E+32}$ & $\num{9.675E+34}$ & $\num{9.675E+38}$ \\
                $n$ & $\num{1.E+10}$ & $\num{6.E+11}$ & $\num{3.6E+13}$ & $\num{8.64E+14}$ & $\num{2.592E+16}$ & $\num{3.11E+17}$ & $\num{3.11E+19}$ \\
                $n \log_{2} n$ & $\num{3.522E+08}$ & $\num{1.763E+10}$ & $\num{9.063E+11}$ & $\num{1.957E+13}$ & $\num{5.299E+14}$ & $\num{5.936E+15}$ & $\num{5.283E+17}$ \\
                $n^{2}$ & $\num{1.E+05}$ & $\num{7.746E+05}$ & $\num{6.E+06}$ & $\num{2.939E+07}$ & $\num{1.61E+08}$ & $\num{5.577E+08}$ & $\num{5.577E+09}$ \\
                $n^{3}$ & $\num{2.154E+03}$ & $\num{8.434E+03}$ & $\num{3.302E+04}$ & $\num{9.524E+04}$ & $\num{2.959E+05}$ & $\num{6.775E+05}$ & $\num{3.145E+06}$ \\
                $2^{n}$ & $\num{3.322E+01}$ & $\num{3.913E+01}$ & $\num{4.503E+01}$ & $\num{4.962E+01}$ & $\num{5.452E+01}$ & $\num{5.811E+01}$ & $\num{6.475E+01}$ \\
                $n!$ & $\num{1.3E+01}$ & $\num{1.5E+01}$ & $\num{1.6E+01}$ & $\num{1.7E+01}$ & $\num{1.8E+01}$ & $\num{1.9E+01}$ & $\num{2.1E+01}$ \\
                \bottomrule
            \end{tabular}
        \end{table}
        \item The ``find-min'' operation is fastest on the binary and Fibonacci heaps, since it is constant. The ``delete-min'' operation should be fastest on the Fibonacci heap, since an $\mathcal{O}(\log n)$ time complexity represents an amortised asymptotic upper bound, whereas the other heaps' $\Theta(\log n)$ is tight. Indeed, given a worst case scenario, both operations take exactly the same time to complete, but otherwise the Fibonacci heap does not take $\log n$ steps to finish the operation in average. The ``insert'' operation is fastest on the binomial and Fibonacci heaps. Thus, the Fibonacci heap is overall the fastest data structure when compared to binary and binomial heaps.
        \item
        \begin{enumerate}
            \item $(\ln n)^{\ln n} \in \Omega(\frac{n}{\ln n})$, for $n_{0} > 7$ and $c = 1$.
            \item $n^{2} \in \Omega(n^{\frac{1}{2}})$, for $n_{0} > 1$ and $c = 1$.
            \item $n! \in \Omega(n^{n - 47})$, for $n_{0} > 1$ and $c = 1$.
            \item $2^{{(\ln n)}^{2}} \in \Omega((\ln n)^{\ln n})$, for $n_{0} > 1$ and $c = 1$.
            \item $n^{2} \in \Theta(n + n^{2})$, for $n_{0} > 1, c_{1} = \frac{1}{2}$ and $c_{2} = 1$.
            \item $n + (\ln n)^{2} \in \Theta(10n + \ln n)$, for $n_{0} > \frac{1}{2}, c_{1} = \frac{1}{2}$ and $c_{2} = 1$.
            \item $\ln 2n \in \Theta(\ln 3n)$, for $n_{0} > 1, c_{1} = \frac{1}{2}$ and $c_{2} = 1$.
            \item $n (\ln n)^{2} \in \mathcal{O}(\frac{n^{2}}{\ln n})$, for $n_{0} > 7$ and $c = 1$.
        \end{enumerate}
    \end{enumerate}
    \item
    \begin{enumerate}
        \item Let $P(n) = \sum_{i = 1}^{n} i = \frac{n (n + 1)}{2}$. $P(1)$ holds, since $\frac{1 (1 + 1)}{2} = 1$. The induction step is proved below, using the hypothesis that $P(k)$ is true.
        \begin{align*}
            P(k) + (k + 1) &\stackrel{?}{=} P(k + 1) \\
            \frac{k (k + 1)}{2} + (k + 1) &\stackrel{?}{=} \\
            \frac{k (k + 1) + 2(k + 1)}{2} &\stackrel{?}{=} \\
            \frac{(k + 1)(k + 2)}{2} &\stackrel{?}{=} \\
            \frac{(k + 1)((k + 1) + 1)}{2} &= P(k + 1).
        \end{align*}
        \item Let $P(n) = \sum_{i = 0}^{n} i^{2} = \frac{n (n + 1) (2n + 1)}{6}$. $P(0)$ holds, since $\frac{0 (0 + 1) (2 \cdot 0 + 1)}{2} = 0$. The induction step is proved below, using the hypothesis that $P(k)$ is true.
        \begin{align*}
            P(k) + (k + 1)^{2} &\stackrel{?}{=} P(k + 1) \\
            \frac{k (k + 1) (2k + 1)}{6} + (k + 1)^{2} &\stackrel{?}{=} \\
            \frac{k (k + 1) (2k + 1) + 6(k + 1)^{2}}{6} &\stackrel{?}{=} \\
            \frac{(k + 1)(k(2k + 1) + 6(k + 1))}{6} &\stackrel{?}{=} \\
            \frac{(k + 1)(2k^2 + 7k + 6)}{6} &\stackrel{?}{=} P(k + 1) \\
            &\stackrel{?}{=} \frac{(k + 1)((k + 1) + 1)(2(k + 1) + 1)}{6} \\
            \frac{(k + 1)(2k^2 + 7k + 6)}{6} &= \frac{(k + 1)(k + 2)(2k + 3)}{6}.
        \end{align*}
        \item Let $P(n) = \sum_{i = 1}^{n} (2i - 1) = n^{2}$. $P(1)$ holds, since $2 \cdot 1 - 1 = 1^{2}$. The induction step is proved below, using the hypothesis that $P(k)$ is true.
        \begin{align*}
            P(k) + 2(k + 1) - 1 &\stackrel{?}{=} P(k + 1) \\
            k^{2} + 2(k + 1) - 1 &\stackrel{?}{=} \\
            k^{2} + 2k + 1 &\stackrel{?}{=} \\
            (k + 1)^{2} &= P(k + 1).
        \end{align*}
        \item Let $P(n) = \sum_{i = 0}^{n} i^{3} = \frac{n^{2} (n + 1)^{2}}{4}$. $P(0)$ holds, since $\frac{0^{2} \cdot (0 + 0)^{2}}{4} = 0$.
        \begin{align*}
            P(k) + (k + 1)^{3} &\stackrel{?}{=} P(k + 1) \\
            \frac{k^{2} (k + 1)^{2}}{4} + (k + 1)^{3} &\stackrel{?}{=} \\
            \frac{k^{2} (k + 1)^{2} + 4(k + 1)^{3}}{4} &\stackrel{?}{=} \\
            \frac{(k^{2} + 4(k + 1))(k + 1)^{2}}{4} &\stackrel{?}{=} \\
            \frac{(k^{2} + 4k + 4)(k + 1)^{2}}{4} &\stackrel{?}{=} \\
            \frac{(k + 2)^{2}(k + 1)^{2}}{4} &= P(k + 1).
        \end{align*}
    \end{enumerate}
    \item
    \begin{enumerate}
        \item Let \[
            T(n) = \begin{cases}
                0, &\text{if } n = 1, \\
                T(n - 1) + c, &\text{if } n > 1
            \end{cases}
        \] for any constant $c$. By the iteration method,
        \begin{align*}
            T(n) &= c + T(n - 1) \\
            &= c + c + T(n - 2) \\
            &= c + c + c + T(n - 3) \\
            &= 
        \end{align*}
    \end{enumerate}
\end{enumerate}

\appendix

\section{Rationale for Q2a}

Define $t_{1}, \dots, t_{7}$ and $s = 10^{10}$ as above. The equation $t_{i} = \frac{f(n)}{s}$, or its simpler form $n = f^{-1}(t_{i} s)$, has to be solved in order to find out the maximum $n$ allowed to execute for a period of time using $f(n)$. Deducing $f^{-1}$ is not trivial for some of the rows. In the case of $n \log_{2} n = c$ for some constant $c$, the Lambert $W$-function is used:
\begin{align*}
    n \log_{2} n &= c \\
    n \ln n &= c \ln 2 \\
    \euler^{\ln n} \ln n &= c \ln 2 \\
    \ln n &= W(c \ln 2) \\
    n &= \euler^{W(c \ln 2)}.
\end{align*}
Moreover, to invert the factorial function, some definitions from~\cite{Cantrell:200110:misc} are used to construct a strict inverse if the input is an integer. Consider the single positive zero of the digamma function $\psi_{0} \approx 1.46163214496836$ and the gamma function $\Gamma(x)$. The inverse factorial function is equal to
\begin{align*}
    \iota(n) = \left[\frac{\ln\frac{n + c}{\sqrt{2 \pi}} - 1}{W(\frac{1}{e}\ln\frac{n + c}{\sqrt{2 \pi}} - 1)} + \frac{1}{2}\right] - 1, \qquad c = \frac{1}{e} \sqrt{2 \pi} - \Gamma(\psi_{0}).
\end{align*}
The table below shows the exact values for all $n = f^{-1}(t_{i} s)$.
\begin{table}[htbp]
    \renewcommand{\arraystretch}{1.2}
    \setlength{\tabcolsep}{7pt}
    \centering
    \begin{tabular}{l*{7}{r}}
        \toprule
            $f(n)$ / $t$ & 1 second & 1 minute & 1 hour & 1 day & 1 month & 1 year & 1 century \\ \midrule
            $\log_{2} n$ & $2^{t_{1} s}$ & $2^{t_{2} s}$ & $2^{t_{3} s}$ & $2^{t_{4} s}$ & $2^{t_{5} s}$ & $2^{t_{6} s}$ & $2^{t_{7} s}$ \\
            $\sqrt{n}$ & $(t_{1} s)^{2}$ & $(t_{2} s)^{2}$ & $(t_{3} s)^{2}$ & $(t_{4} s)^{2}$ & $(t_{5} s)^{2}$ & $(t_{6} s)^{2}$ & $(t_{7} s)^{2}$ \\
            $n$ & $t_{1} s$ & $t_{2} s$ & $t_{3} s$ & $t_{4} s$ & $t_{5} s$ & $t_{6} s$ & $t_{7} s$ \\
            $n \log_{2} n$ & $\euler^{W(t_{1} s \ln 2)}$ & $\euler^{W(t_{2} s \ln 2)}$ & $\euler^{W(t_{3} s \ln 2)}$ & $\euler^{W(t_{4} s \ln 2)}$ & $\euler^{W(t_{5} s \ln 2)}$ & $\euler^{W(t_{6} s \ln 2)}$ & $\euler^{W(t_{7} s \ln 2)}$ \\
            $n^{2}$ & $\sqrt{t_{1} s}$ & $\sqrt{t_{2} s}$ & $\sqrt{t_{3} s}$ & $\sqrt{t_{4} s}$ & $\sqrt{t_{5} s}$ & $\sqrt{t_{6} s}$ & $\sqrt{t_{7} s}$ \\
            $n^{3}$ & $\sqrt[3]{t_{1} s}$ & $\sqrt[3]{t_{2} s}$ & $\sqrt[3]{t_{3} s}$ & $\sqrt[3]{t_{4} s}$ & $\sqrt[3]{t_{5} s}$ & $\sqrt[3]{t_{6} s}$ & $\sqrt[3]{t_{7} s}$ \\
            $2^{n}$ & $\log_{2}(t_{1} s)$ & $\log_{2}(t_{2} s)$ & $\log_{2}(t_{3} s)$ & $\log_{2}(t_{4} s)$ & $\log_{2}(t_{5} s)$ & $\log_{2}(t_{6} s)$ & $\log_{2}(t_{7} s)$ \\
            $n!$ & $\iota(t_{1} s)$ & $\iota(t_{2} s)$ & $\iota(t_{3} s)$ & $\iota(t_{4} s)$ & $\iota(t_{5} s)$ & $\iota(t_{6} s)$ & $\iota(t_{7} s)$ \\
        \bottomrule
    \end{tabular}
\end{table}

\bibliographystyle{alpha}
{\footnotesize
\bibliography{ref}}

\end{document}

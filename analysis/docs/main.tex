\documentclass[12pt]{article}

\usepackage[T1]{fontenc}
\usepackage[a4paper, margin=1.5cm]{geometry}
\usepackage[colorlinks, urlcolor=blue, citecolor=red]{hyperref}
\usepackage[utf8]{inputenc}
\usepackage{amsmath, amsfonts, booktabs, enumitem, parskip}

\newcommand{\euler}{\mathrm{e}}

\begin{document}

\textsc{Graduate Program in Computer Science,
  Universidade Federal de Santa Catarina} \\
\textsc{INE410104 (Design and Analysis of Algorithms)}

\textsc{Solutions to the 1\textsuperscript{st} Set of Exercises} \\
\textsc{Gustavo Zambonin, Matheus S. P. Bittencourt}

\section{Answers}

\begin{enumerate}
    \item 
    \begin{enumerate}
        \item Let the functions that represent the time complexities of algorithms $A$ and $B$ be $f(n) = n^{3}$ and $g(n) = 128n^{2}$, respectively, with $n \in \mathbb{N}$. Further, consider $t(n) = \frac{f(n)}{g(n)} \Rightarrow \frac{n}{128}$. By the ratio $t(n)$, algorithm $A$ is faster than $B$ when $n > 128$.
        \item Let the old and new computers be named $N_{1}$ and $N_{2}$, respectively, and $f(n) = 2^{n}$. The equation $t = \frac{2^{N_{1}}}{s}$ gives the time spent to solve an instance of $f(n)$ such that $N_{1}$ executes $s$ instructions per second. Analogously, let $t = \frac{2^{N_{2}}}{20s}$ represent the previous equation for the new computer. Solving for $N_{2}$ in terms of $N_{1}$:
        \begin{align*}
            \frac{2^{N_{1}}}{s} = \frac{2^{N_{2}}}{20s} \\
            2^{N_{2}} = 20 \cdot 2^{N_{1}} \\
            N_{2} = N_{1} + \log_{2} 20 \\
            N_{2} \approx N_{1} + 4.32.
        \end{align*}
        Ergo, one cannot execute $f(2n)$ and expect comparable results on $N_{2}$, since $N_{2} \ll 2 N_{1}$.
    \end{enumerate}
    \item
    \begin{enumerate}
        \item 
        \item The ``find-min'' operation is fastest on the binary and Fibonacci heaps, since it is constant. The ``delete-min'' operation should be fastest on the Fibonacci heap, since an $\mathcal{O}(\log n)$ time complexity represents an amortised asymptotic upper bound, whereas the other heaps' $\Theta(\log n)$ is tight. Indeed, given a worst case scenario, both operations take exactly the same time to complete, but otherwise the Fibonacci heap does not take $\log n$ steps to finish the operation in average. The ``insert'' operation is fastest on the binomial and Fibonacci heaps. Thus, the Fibonacci heap is overall the fastest data structure when compared to binary and binomial heaps.
    \end{enumerate}
\end{enumerate}

\appendix

\section{Rationale for Q2a}

Let $t_{1}, \dots, t_{7}$ be the number of seconds in each period of the header row and $s = 10^{10}$. The equation $t_{i} = \frac{f(n)}{s}$ for $n$ has to be solved in order to find out the maximum $n$ allowed to execute for a period of time using $f(n)$, \emph{i.e.} $n = f^{-1}(t_{i} s)$. Deducing $f^{-1}$ is not trivial for some of the rows. In the case of $n \log_{2} n = c$ for some constant $c$, the Lambert $W$-function is used:
\begin{align*}
    n \log_{2} n &= c \\
    n \ln n &= c \ln 2 \\
    \euler^{\ln n} \ln n &= c \ln 2 \\
    \ln n &= W(c \ln 2) \\
    n &= \euler^{W(c \ln 2)}.
\end{align*}
Moreover, to invert the factorial function some definitions from~\cite{Cantrell:200110:misc} are used to construct a strict inverse, if the input is an integer. Consider the single positive zero of the digamma function $\psi_{0} \approx 1.46163214496836$ and the gamma function $\Gamma(x)$. The inverse factorial function is equal to
\begin{align*}
    i(n) = \left[\frac{\ln\frac{n + c}{\sqrt{2 \pi}} - 1}{W(\frac{1}{e}\ln\frac{n + c}{\sqrt{2 \pi}} - 1)} + \frac{1}{2}\right] - 1, \qquad c = \frac{1}{e} \sqrt{2 \pi} - \Gamma(\psi_{0}).
\end{align*}

The table below shows the exact values for $n = f^{-1}(t_{i} s)$.
\begin{table}[htbp]
    \renewcommand{\arraystretch}{1.2}
    \setlength{\tabcolsep}{7pt}
    \centering
    \begin{tabular}{l*{7}{r}}
        \toprule
            $f(n)$ / $t$ & 1 second & 1 minute & 1 hour & 1 day & 1 month & 1 year & 1 century \\ \midrule
            $\log_{2} n$ & $2^{t_{1} s}$ & $2^{t_{2} ss}$ & $2^{t_{3} s}$ & $2^{t_{4} s}$ & $2^{t_{5} s}$ & $2^{t_{6} s}$ & $2^{t_{7} s}$ \\
            $\sqrt{n}$ & $(t_{1} s)^{2}$ & $(t_{2} s)^{2}$ & $(t_{3} s)^{2}$ & $(t_{4} s)^{2}$ & $(t_{5} s)^{2}$ & $(t_{6} s)^{2}$ & $(t_{7} s)^{2}$ \\
            $n$ & $t_{1} s$ & $t_{2} s$ & $t_{3} s$ & $t_{4} s$ & $t_{5} s$ & $t_{6} s$ & $t_{7} s$ \\
            $n \log_{2} n$ & $\euler^{W(t_{1} s \ln 2)}$ & $\euler^{W(t_{2} s \ln 2)}$ & $\euler^{W(t_{3} s \ln 2)}$ & $\euler^{W(t_{4} s \ln 2)}$ & $\euler^{W(t_{5} s \ln 2)}$ & $\euler^{W(t_{6} s \ln 2)}$ & $\euler^{W(t_{7} s \ln 2)}$ \\
            $n^{2}$ & $\sqrt{t_{1} s}$ & $\sqrt{t_{2} s}$ & $\sqrt{t_{3} s}$ & $\sqrt{t_{4} s}$ & $\sqrt{t_{5} s}$ & $\sqrt{t_{6} s}$ & $\sqrt{t_{7} s}$ \\
            $n^{3}$ & $\sqrt[3]{t_{1} s}$ & $\sqrt[3]{t_{2} s}$ & $\sqrt[3]{t_{3} s}$ & $\sqrt[3]{t_{4} s}$ & $\sqrt[3]{t_{5} s}$ & $\sqrt[3]{t_{6} s}$ & $\sqrt[3]{t_{7} s}$ \\
            $2^{n}$ & $\log_{2}(t_{1} s)$ & $\log_{2}(t_{2} s)$ & $\log_{2}(t_{3} s)$ & $\log_{2}(t_{4} s)$ & $\log_{2}(t_{5} s)$ & $\log_{2}(t_{6} s)$ & $\log_{2}(t_{7} s)$ \\
            $n!$ & \\
        \bottomrule
    \end{tabular}
    \label{tab:1}
\end{table}

\bibliographystyle{alpha}
{\footnotesize
\bibliography{ref}}

\end{document}

\documentclass[12pt]{article}

\usepackage[T1]{fontenc}
\usepackage[a4paper, margin=1.5cm]{geometry}
\usepackage[colorlinks, urlcolor=blue, citecolor=red]{hyperref}
\usepackage[utf8]{inputenc}
\usepackage{amsmath, amsfonts, enumitem, parskip}

\begin{document}

\textsc{Graduate Program in Computer Science,
  Universidade Federal de Santa Catarina} \\
\textsc{INE410104 (Design and Analysis of Algorithms)}

\textsc{Solutions to the 2\textsuperscript{nd} Set of Exercises} \\
\textsc{Emmanuel Podestá Jr., Gustavo Zambonin, Matheus S. P. Bittencourt}

\section{Answers}

\begin{enumerate}
  \item
  \begin{enumerate}
    \item 
    \item 
  \end{enumerate}
  \item
  \begin{enumerate}
    \item 
    \item 
  \end{enumerate}
  \item
  \begin{enumerate}
    \item 
    \item 
  \end{enumerate}
  \item 
  \begin{enumerate}
    \item A decision problem $p \in \mathbf{NP}$ is called \textbf{NP}-complete if every decision problem in \textbf{NP} has a polynomial-time, many-one reduction to $p$. The travelling salesman problem in its decision version (TSPd) is an example of a \textbf{NP}-complete problem. Therefore, if one were to give a polynomial time algorithm to solve TSPd, and considering that it is possible to reduce all problems in \textbf{NP} to an instance of TSPd, one could solve these in polynomial time. In other words, \textbf{P} would be equal to \textbf{NP}.
    \item Recall that a problem $p$ is \textbf{NP}-hard if there exists a polynomial-time, many-one reduction from a \textbf{NP}-complete problem to $p$. Usually, it is said that \textbf{NP}-hard problems are ``at least as hard as the hardest problems in \textbf{NP}'', \emph{i.e.} those that are \textbf{NP}-complete. If a polynomial reduction from a problem $q$ to $p$ exists, then by the definition above, $q$ is indeed 
    \item
    \item
  \end{enumerate}
\end{enumerate}

\end{document}
\documentclass[12pt]{article}

\usepackage[T1]{fontenc}
\usepackage[a4paper, margin=1.5cm]{geometry}
\usepackage[colorlinks, urlcolor=blue, citecolor=red]{hyperref}
\usepackage[utf8]{inputenc}
\usepackage{amsmath, amsfonts, enumitem, parskip}

\begin{document}

\textsc{Graduate Program in Computer Science,
  Universidade Federal de Santa Catarina} \\
\textsc{INE410104 (Design and Analysis of Algorithms)}

\textsc{Solutions to the 2\textsuperscript{nd} Set of Exercises} \\
\textsc{Emmanuel Podestá Jr., Gustavo Zambonin, Matheus S. P. Bittencourt}

\section{Answers}

\begin{enumerate}
  \item
  \begin{enumerate}
    \item 
    \item It is known that Dijkstra's algorithm returns the shortest path between two vertices in a graph $G = (V, E)$. The description of the input considers the function $w : V \to \mathbb{R}^{+}$
  \end{enumerate}
  \item
  \begin{enumerate}
    \item 
    \item 
  \end{enumerate}
  \item
  \begin{enumerate}
    \item 
    \item 
  \end{enumerate}
  \item 
  \begin{enumerate}
    \item A decision problem $p \in \mathbf{NP}$ is called \textbf{NP}-complete if every decision problem in \textbf{NP} has a polynomial-time, many-one reduction to $p$. The travelling salesman problem in its decision version (TSPd) is an example of a \textbf{NP}-complete problem. Therefore, if one were to give a polynomial time algorithm to solve TSPd, and considering that it is possible to reduce all problems in \textbf{NP} to an instance of TSPd, one could solve these in polynomial time. In other words, \textbf{P} would be equal to \textbf{NP}.
    \item Recall that a problem $p$, not necessarily in \textbf{NP}, is \textbf{NP}-hard if there exists a polynomial-time, many-one reduction from a \textbf{NP}-complete problem to $p$. Usually, it is said that \textbf{NP}-hard problems are ``at least as hard as the hardest problems in \textbf{NP}'', \emph{i.e.} those that are \textbf{NP}-complete. If a polynomial reduction from a problem $q$ to $p$ exists, then by the definition above, $q$ is in the hardest class of problems, and may not even be solvable.
    \item Solving and verifying a problem in a quantity of time polynomial to the size of the input are, respectively, the informal definitions to the complexity classes \textbf{P} and \textbf{NP}. Problems in the latter class have no known polynomial time algorithms to solve them, but given a possible solution, there exists an algorithm that can decide if it is valid or not. For example, factoring a large integer is known to be computationally hard, but given its factors, it is trivial to check for a correct answer.
    \item Approximation algorithms are provably bounded to find solutions within some ``distance'' of solutions to the original problems. This is equivalent to the idea that the approximate solution returned is moderately good, or no worse than the worst case scenario. To the contrary, heuristics present no mathematical proofs that constrain the quality of the solution, often requiring trade-offs to work efficiently or accurately.
  \end{enumerate}
\end{enumerate}

\end{document}